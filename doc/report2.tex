\documentclass[a4paper,12pt]{article}
\usepackage{CodeReport}

\begin{document}


\begin{center} % Everything within the center environment is centered.
	{\Large \bf Coding Report 2} % <---- Don't forget to put in the right number
	\vspace{2mm}
\end{center}  

\vspace{0.4cm}


\section{Problem Description}
In this coding report,
we explore the implementation and performance of gaussian elimination and LU factorization.
We first demonstrate the methods by solving linear systems from homework 2
$$
\begin{bmatrix}
	2 & 4 & 5 \\
	7 & 6 & 5 \\
	9 & 11 & 3
\end{bmatrix}
x = 
\begin{bmatrix}
	3 \\
	2 \\
	1
\end{bmatrix}.
$$

Then we compare two methods:
\begin{enumerate}
	\item gaussian elimination solve with back substitution (GE)
	\item LU factorization then solve with forward and back substitution (LU)
\end{enumerate}
on size $n = 50, 100, 250, 500$ matrices.
The linear system $Ax = b$ to solve will have the following setup
$$
A = 5\times I + R
$$
where random entries $r_{ij}, b_{ij} \sim N(0, 1)$.
We record the errors of two methods in a table and then plot their execution time v.s. $n$.

\section{Results}

\subsection{Homework Problem}
By using LU factorization method on $A$, we get
$$
L = \begin{bmatrix}
	1 & 0 & 0 \\
	3.5 & 1 & 0 \\
	4.5 & 0.875 & 1 
\end{bmatrix} \\
U = \begin{bmatrix}
	2 & 4 & 5 \\
	0 & -8 & -12.5 \\
	0 & 0 & -8.5625
\end{bmatrix}
$$
which are the same as the results from homework.

By calling `gauss\_solve' and `lu\_solve',
both methods get the same result for 8 digits
$$
x = \begin{bmatrix}
	-& 0.25547445 \\
	 & 0.13868613 \\
	 & 0.59124088
\end{bmatrix}.
$$

\subsection{Comparison}
To make the comparison fairer,
for each $n$ we call `gauss\_solve' and `lu\_solve' $10$ times
and record the average error and execution time.

\begin{table}[H]
\begin{center}
	\begin{tabular}{lrrr}
	\toprule
	{}  &            GE &            LU \\
	\midrule
	 50 &  1.621020e-11 &  1.602339e-11 \\
	100 &  1.292846e-11 &  7.911123e-12 \\
	250 &  8.170815e-10 &  7.128272e-10 \\
	500 &  1.466036e-09 &  2.159379e-09 \\
	\bottomrule
	\end{tabular}
	\caption{Errors of $\| Ax - b\|_{2}$ for GE and LU methods}
\end{center}
\end{table}

\begin{figure}[H]
    \centering
     \includegraphics[width=0.7\textwidth]{img/report2_cell_10_output_1.png}
     \caption{Execution time to solve $Ax=b$ for GE and LU v.s. $n$}
     \label{fig:1}   
\end{figure}

From the above table and graph, we can see that the solution produced by LU
has larger errors than using gaussian elimination and back substitution directly.
However, when the size of the matrix goes up, LU solve is faster than ge solver.

\subsection{Experiments with matrix family $\hat{A} = R$}

\begin{table}[H]
\begin{center}	
	\begin{tabular}{lrrr}
	\toprule
	{} &             GE &            LU \\
	\midrule
	 50 &  3.564803e-11 &  3.145187e-11 \\
	100 &  4.906939e-11 &  2.390745e-11 \\
	250 &  1.890665e-09 &  1.223007e-09 \\
	500 &  1.799276e-09 &  1.455843e-09 \\
	\bottomrule
	\end{tabular}
	\caption{Errors of $\| \hat{A}x - b\|_{2}$ for GE and LU methods}
\end{center}
\end{table}

\begin{figure}[H]
    \centering
     \includegraphics[width=0.7\textwidth]{img/report2_cell_15_output_1.png}
     \caption{Execution time to solve $\hat{A}x=b$ for GE and LU v.s. $n$}
     \label{fig:2}   
\end{figure}

In this case,
we can see that errors of LU is smaller than GE,
and LU is still faster.

\section{Collaboration}
No collaboration on this project.


\section{Academic Integrity}
On my personal integrity as a student and member of the UCD community, I have not given nor received any unauthorized assistance on this assignment.


\section{Appendix}
Note that there are two source files used for this report,
each's name are stated in the title of the file listing.
For the following code, please use \textbf{python3.10} and have the following packages installed
\begin{itemize}
	\item numpy
	\item returns.
\end{itemize}

\lstinputlisting[language=Python, title=\url{numerical_methods/linear_direct_methods.py}]{../numerical_methods/linear_direct_methods.py}
\lstinputlisting[language=Python, title=main.py]{report2.py}

\end{document}