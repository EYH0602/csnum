\documentclass[a4paper,12pt]{article}
\usepackage{CodeReport}

\begin{document}


\begin{center} % Everything within the center environment is centered.
	{\Large \bf Coding Report 4} % <---- Don't forget to put in the right number
	\vspace{2mm}
\end{center}  


\section{Problem Description}

Then we compare four methods: Jacobi, Gauss-Seidel, SOR with $\omega = 1.1$ (sor11), and SOR $\omega = 0.9$ (sor09).
The test will be performed random matrices of size $n = 10, 25, 50, 100, 200, 500$.
The linear system $Ax = b$ to solve will have the following setup
$$
A = \frac{n}{2} \times I + R
$$
where random entries $r_{ij}, b_{ij} \sim N(0, 1)$.
We plot their execution time and number of iteration used v.s. $n$.
Finally, we will record errors at every step for $n = 500$ 
and plot it with respect to number of iteration.

\section{Results}

\subsection{Homework Problem}
Let 
$$
A = \begin{bmatrix}
	4 & 1 & -1 \\
	-1 & 3 & 1 \\
	2 & 2 & 6
\end{bmatrix},
b = \begin{bmatrix}
	5 \\ -4 \\ 1
\end{bmatrix}
$$
the code gives solutions
$$
x_j = \begin{bmatrix}
	 1.54166667 \\
	-0.85648148 \\
	-0.05092593 
\end{bmatrix},
x_g = \begin{bmatrix}
	 1.45196759 \\
	-0.83391204 \\
	-0.03935185 
\end{bmatrix},
x_s = \begin{bmatrix}
	 1.43231701 \\
	-0.83290758 \\
	-0.02942417 
\end{bmatrix},
$$
which are the same as the results from homework if we round the solution to 3 digits.


\subsection{Comparison}

We compare the time and iteration used for each methods until the relative update falls below $10^{-8}$.
\begin{figure}[H]
    \centering
    \includegraphics[width=0.55\textwidth]{img/report3_cell_11_output_1.png}
    \caption{wall clock time v.s. $n$ size of matrix}
    \label{fig:0}   
\end{figure}

For execution time,
we can see that Gauss-Seidel is the fastest for each $n$,
followed by Jacobi and two SOR methods.
Gauss-Seidel is generally a order faster than Jacobi,
and SOR behaves similar for $\omega = 1.1, 0.9$.

\begin{figure}[H]
    \centering
    \includegraphics[width=0.55\textwidth]{img/report3_cell_13_output_1.png}
    \caption{number of iterations v.s. $n$ size of matrix}
    \label{fig:1}   
\end{figure}

For number of iteration,
we can see that Gauss-Seidel is still the lowest.
When $n$ is small (i.e. $\leq 100$). SOR methods uses less iteration than Jacobi,
but Jacobi uses lower number of iteration when the size goes up.
Two SOR methods still behaves similarly and not ideally,
this might be caused by bad choices by $\omega$.
One thing to note is that with larger $n$,
it requires less number of iteration to converge for all methods.

\begin{figure}[H]
    \centering
    \includegraphics[width=0.55\textwidth]{img/report3_cell_15_output_1.png}
    \caption{Error of each methods at $n = 500$ v.s. number of iteration}
    \label{fig:2}   
\end{figure}

Errors at each step also shows a similar trend that Gauss-Seidel is the lowest,
followed by Jacobi and SOR.
And the differences in errors becomes hard to distinguish after about 4 iterations.

For $\hat{A} = randn(n, n)$ matrices,
sometimes Jacobi and Gauss-Seidel failed to converge since $\hat{A}$ may not be diagonally dominant.


\section{Collaboration}
No collaboration on this project.


\section{Academic Integrity}
On my personal integrity as a student and member of the UCD community, I have not given nor received any unauthorized assistance on this assignment.


\section{Appendix}
\lstinputlisting[language=Python, title=\url{numerical_methods/linear_iter_methods.py}]{../numerical_methods/linear_iter_methods.py}
\lstinputlisting[language=Python, title=main.py]{report3.py}

\end{document}